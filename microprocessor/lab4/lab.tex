\documentclass{bmstu}
\usepackage{xparse}
\usepackage{mathtools}
\begin{document}

% Аргументы, помеченные как необязательные, могут быть пустыми. В таком случае соответствующее этому аргументу поле (например, ФИО консультанта) добавлено не будет.

% Титульная страница

% Отчет

\titlespacing*{\chapter}{0pt}{5pt}{0px}

\makeatletter %%%%% <---- Starting chapter without a pagebreak
\renewcommand\chapter{\par%
\thispagestyle{plain}% \global\@topnum\z@
\@afterindentfalse \secdef\@chapter\@schapter}
\makeatother

\def\numberWork{1}


\makereporttitle
    {Информатика и системы управления} % Название факультета
    {Компьютерные системы и сети} % Название кафедры
    {09.03.01 Информатика и вычислительная техника} % направение
    {лабораторной работе №~4} % Название работы (в дат. падеже)
    {Таймеры микроконтроллеров ATx8515.} % Тема работы
    {Микропроцессорные системы.} % Название курса (необязательный аргумент)
    {} % Номер варианта (необязательный аргумент)
    {ИУ6-62Б} % Номер группы
    {
    	{ИУ6-62Б}
    	{А.Е.Медведев} % ФИО студента
    	{Б.И.Бычков} % ФИО преподавателя
    } 
    
\chapter{Цель работы:}

\begin{enumerate}
\item[---] изучение структур и режимов работы таймеров и их программирование, 
\item[---] анализ схем включения таймеров для проведения исследований, 
\item[---] программирование задач с таймером.
\end{enumerate}

\chapter{Задание 1.} 

Проверить на плате STK500 работу исходной программы 4.1. 
Изменить программу, исключив влияние на работу таймера возможность “дребезга” кнопки. Для этого запрограммировать 
линию порта PB0 на вывод для программного ввода в таймер положительного сигнала при замыкании кнопки. Формирование 
сигнала «0/1» для таймера выполнить, предварительно проверив состояние кнопки командой sbiс PINx,b (порт и линию 
для кнопки выбрать самостоятельно, настроив ее на ввод и подключив подтягивающий резистор). При замыкании кнопки 
(сигнале «0») программно сформировать сигнал «0/1» на выходе PB0, в противном случае процессор остается в цикле 
ожидания. 

Так как новый ввод в таймер можно производить только после отпускания кнопки (сигнала «1»), в программу следует 
добавить команды задержки rcall DELAY и проверки отпускания кнопки sbis PINx,b. (Примечание. При программном 
управлении работой таймера режим sleep должен быть отменен). 

Перед загрузкой программы выполнить ее отладку в AVR Studio. Убедившись в правильности работы программы, проверить 
ее на плате STK500. 

Код программы представлен в листинге \ref{lst:lab4_1.txt}

\includelisting
	{lab4_1.txt}
	{Листинг первой программы}

На рисунках \ref{img:struct_scheme_of_timer.png} --- \ref{img:struct_scheme_of_timer_t1.png} изображены структурные схемы таймеров/счетчиков Т0 и Т1.

\includeimage
	{struct_scheme_of_timer.png}
	{f}
	{H}
	{0.7\textwidth}
	{Структурная схема таймера/счетчика Т0}

\includeimage
	{struct_scheme_of_timer_t1.png}
	{f}
	{H}
	{0.7\textwidth}
	{Структурная схема таймера/счетчика Т1}

\chapter{Задание 2.}

Проверить работу программы 4.2. Оценить время свечения светодиодов при нажатии кнопки SW0 и при нажатии кнопки SW1 и сравнить его с расчетным значением. Изменив настройки таймера, уменьшить вдвое время включения светодиодов.
 
Расчетные значения:

$T = (65536 - TCNT1) * \frac{K}{F_{ck}}, где F_{ck} = 3.69МГц, K = \frac{K\_SW0}{K\_SW1}(\frac{1024}{256})$,

TCNT1 – начальное значение счетчика = 32786.

Рассчитаем время для T0 и T1. \\
$T0 = (65536 - 32 768) * \frac{1024}{(3.69 * 10^6)} = 9.09 = 9.1с. $ \\
$T1 = (65536 - 32 768) * \frac{256}{(3.69 * 10^6)} = 2.27 = 2.3с. $

Начальные параметры для задания 4.2:

\begin{table}[H]
\caption{Таблица сранения времени без изменений}
\label{table:eq_table_1}
\begin{tabular}{| c | c |}
	\hline
		Расчетное время, с & Практическое время, с \\
	\hline
		9,1 & 10,2 \\
	\hline
		2,3 & 2,7 \\
	\hline
\end{tabular}
\end{table}

Измененные параметры для задания 4.2, в 2 раза меньше исходного: \\ 
$T0 = (65536 - 49152) * \frac{1024}{(3.69 * 10^6)} = 4,55с. $ \\
$T1 = (65536 - 49152) * \frac{256}{(3.69 * 10^6)} = 1,12с$


\begin{table}[H]
\caption{Таблица сранения времени c изменений}
\label{table:eq_table_2}
\begin{tabular}{| c | c |}
	\hline
		Расчетное время, с & Практическое время, с \\
	\hline	
		4,55 & 4,7 \\
	\hline
		1,12 & 1,2 \\
	\hline
\end{tabular}
\end{table}

\includelisting
	{lab4_2.txt}
	{Листинг второй программы}

\chapter{Задание 3.}

Проверить работу программы 4.3. Результат работы программы должен соответствовать диаграммам на рис.3. При нажатии 
на кнопку SW0 светодиоды работают в следующей последовательности: оба светодиода горят, далее выключается LED0, 
затем LED1, включается LED0, затем LED1 и т.д. В любой момент процесс можно остановить нажатием кнопки SW2. 

Изменить параметры настройки таймера так, чтобы параметры выходных сигналов соответствовали выбранным значениям из 
ряда: 

длительность tи $\approx$ 2 с, 3 с, 4 с, 5 с, 6 с; 
задержка tз = $ \frac{1}{4} $ tи, 
$\frac{1}{2}$ tи, $\frac{3}{4}$ tи.

Выбранные значения: 

tи $\approx$ 6 с. => tз = $\frac{1}{2}$ tи = 3с. 

Рассчитаем значения регистров по следующим формулам: 

$OCR1A=\frac{tи*Fck}{K}=\frac{6c * 3,69 * 10^6}{1024} \approx 21621= \text{0x5475};$ 

$\frac{tз*Fck}{K} = \frac{3c * 3,69 * 10^6}{1024} \approx 10810= \text{0x2А3А}; $

$tз= (OCR1A - OCR1B) * \frac{K}{Fcк} => OCR1B = 5475 - \text{2А3А} = 2A3B. $

\includelisting
	{lab4_3.txt}
	{Листинг третьей программы}

\chapter{Задание 4.}
Проверить работу программы 4.4 на плате STK500. При нажатии на SW0 или SW1 светодиоды попеременно включаются/
выключаются в соответствии с заданным порогом сравнения F1 или F2. Зарисовать диаграммы включения/выключения 
светодиодов LED0, LED1 с учетом порогов сравнения, указав период и длительности сигналов. 

Выбранный вариант: с коэффициентом заполнения (отношение длительности к периоду) $\frac{1}{4}$ для 9- разрядного режима ШИМ; 

Расчеты для режима F1: \\
$\frac{tu}{T} = \frac{1}{4} $ \\
$T = 2*TOP*Tcnt = 2*TOP* \frac{K}{Fck} = 2*511*\frac{1024}{3.69 * 10^6} = 0,2836c. $ \\
$t(u)=\frac{T}{4} = 0,0709c. $ \\
$t(u) = 2*OCR1A(B) *\frac{K}{Fck} = 0,0709c.$ \\
$OCR1A = 12810 = 0\text{x}80.$\\

На рисунке \ref{img:4.png} изображена диаграмма ШИМ для режима F1.

Расчеты для режима F2: \\
$\frac{tu}{T} = \frac{3}{4} $ \\ 
$OCR1B = \frac{3}{4} * TOP= \frac{3}{4} * 511 = 682 = 0 \text{x} 2AA $ \\

На рисунке \ref{img:4.png} изображена диаграмма ШИМ для режима F2.

\includelisting
	{lab4_4.txt}
	{Код четвёртой программы}

\chapter{Задание 5.} 
Загрузить программу в STK500. Для проверки работы программы включите одновременно секундомер часов и запустите 
программу. После останова программы проверьте показания времени на часах и в регистрах захвата таймера. После 
сброса RESET повторите эксперимент (2-3 раза) и оцените погрешность замеров. 

\includelisting
	{lab4_5.txt}
	{Код пятой программы}



Оценка работы программы 

Проведём эксперименты и сравним расчётное время по таймеру с измеренным с помощью секундомера:

\begin{table}
\label{table:eq_table_2}
\caption{Таблица сравнения времени}
\begin{tabular} {|c|c|c|}
	\hline
		№ &Время с секундомера, с & Расчетное время, с \\
	\hline
		1 & 4,7 & 4,69 \\
	\hline
		2 & 9,1 & 10,18 \\
	\hline
		3 & 3,5 & 3,18 \\
	\hline	
\end{tabular}
\end{table}


Расчеты: 
\begin{enumerate} [label=\arabic*)]
\item $\frac{16957*1024}{3.69*10^6} \approx 4,69$ 
\item $\frac{36698*1024}{3.69*10^6} \approx 10,18$ 
\item $\frac{12142*1024}{3.69*10^6} \approx 3,36$
\end{enumerate}
Погрешность: 
\begin{enumerate}[label=\arabic*)] 
\item $\frac{4,7-4,69}{4,69} * 100\% = 0,002 14$ 
\item $\frac{10,18-9,1}{10,18} * 100\% = 0,11 $
\item $\frac{3,5-3,36}{3,36} * 100\% = 0.041 $
\end{enumerate}

Как можно заметить, погрешности крайне малы для данной задачи.
\newline


\chapter{Вывод:}
В ходе данной работы были изучены структуры и режимы работы таймеров, и их программирование, проанализированы 
схемы включения таймеров для проведения исследований, проведено программирование задач с таймером. Таймер состоит 
из базового счетчика, регистра управления и схемы управления. Также таймер может содержать регистры для хранения 
текущего состояния и сравнения, а также компараторы (устройства сравнения). Были проверены режимы счётчика, 
таймера, широтно-импульсной модуляции, сравнения и захвата. Режим счётчика позволяет считать события. Режим 
таймера даёт возможность формировать временные интервалы. ШИМ режим позволяет генерировать сигналы с 
программируемой частотой и скважностью. Функция сравнения даёт возможность сравнивать содержимое счётчика с 
регистром сравнения, а функция захвата обеспечивает сохранение состояния таймера или счётчика в регистре захвата.

\end{document}
\begin{comment}
На~рисунке~\ref{img:tux} символ семейства Unix-подобных операционных систем Linux.
Он отличается от~<<обычных>> пингвинов желтым цветом клюва и~лап.

\includeimage
    {tux} % Имя файла без расширения (файл должен быть расположен в директории inc/img/)
    {f} % Обтекание (без обтекания)
    {h} % Положение рисунка (см. figure из пакета float)
    {0.25\textwidth} % Ширина рисунка
    {Символ Linux (Tux)} % Подпись рисунка

На~листингах представлен исходный код программы Hello World на~языке программирования C в~двух вариантах оформления.

\includelisting
    {main.c} % Имя файла с расширением (файл должен быть расположен в директории inc/lst/)
    {Исходный код программы Hello World} % Подпись листинга

\includelistingpretty
    {main.c} % Имя файла с расширением (файл должен быть расположен в директории inc/lst/)
    {c} % Язык программирования (необязательный аргумент)
    {Исходный код программы Hello World} % Подпись листинга



\end{comment}
