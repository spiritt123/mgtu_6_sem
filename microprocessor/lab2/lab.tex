\documentclass{bmstu}
\usepackage{xparse}
\usepackage{amssymb}


\makeatletter %%%%% <---- Starting chapter without a pagebreak
\renewcommand\chapter{\par%
\thispagestyle{plain}% \global\@topnum\z@
\@afterindentfalse \secdef\@chapter\@schapter}
\makeatother

\def\numberWork{1}

\usepackage{graphicx}
\begin{document}

\makereporttitle
    {Информатика и системы управления} % Название факультета
    {Компьютерные системы и сети} % Название кафедры
    {09.03.01 Информатика и вычислительная техника} % направение
    {лабораторной работе №~2} % Название работы (в дат. падеже)
    {Обработка внешних прерываний в микроконтроллерах AVR} % Тема работы
    {Микропроцессорные системы} % Название курса (необязательный аргумент)
    {19} % Номер варианта (необязательный аргумент)
    {ИУ6-62Б} % Номер группы
    {
    	{ИУ6-62Б}
    	{А.Е.Медведев} % ФИО студента
    	{Б.И. Бычков} % ФИО преподавателя
    } 
    
\chapter{Цель работы:}
\begin{itemize}
\item[--] изучение системы прерываний микроконтроллеров AVR,
\item[--] освоение системы команд микроконтроллеров AVR,
\item[--] ознакомление с работой стека при вызове подпрограмм и обработчиков прерываний,
\item[--] программирование внешних прерываний.  
\end{itemize}

\chapter{Задания:}
\section{Задание 1}
Запустив AVR Studio, проверить работу программы в шаговом режиме. С целью ускорения отладки сократить время задержек до минимума. Проконтролировать работу стека при вызове подпрограмм delay1, delay2.
\includelisting
	{demo1.asm}
	{Код реализации прерывания на плате ATmega8515}

При вызове подпрограммы или функции в cтек помещется адрес возврата. По условию задания начало стека установлена в конец секции данных. Работы стека продемонстрирована на рисунках \ref{img:demo1_dis.png} --- \ref{img:demo2_1.png}

\includeimage
    {demo1_dis.png} % Имя файла без расширения (файл должен быть расположен в директории inc/img/)
    {f} % Обтекание (без обтекания)
    {H} % Положение рисунка (см. figure из пакета float)
    {1\textwidth} % Ширина рисунка
    {Дизассемблер программы} 

\includeimage
    {demo2_1.png} % Имя файла без расширения (файл должен быть расположен в директории inc/img/)
    {f} % Обтекание (без обтекания)
    {H} % Положение рисунка (см. figure из пакета float)
    {1\textwidth} % Ширина рисунка
    {Состояние стека при пызове подпрограммы или функции} 



\section{Задание 2}
Вносим изменения  и дополнения в исходный текст программы 2.1, касающиеся обработки прерываний. На этапе инициализации указываются область стека для сохранения адресов возврата, при необходимости адреса векторов прерываний и сами векторы, маска прерываний, общее разрешение прерываний. Завершаем инициализацию переводом процессора в фоновый режим ожидания:

\includelisting
	{demo2.asm}
	{Код реализации прерывания на плате ATmega8515}

Состояние регистров и стека продемонстрировано на рисунках \ref{img:demo2_2.png} --- \ref{img:demo2_2_r.png}.
Результат работы программы был отлажен в среде протеус, как показано на рисунке \ref{img:demo2_2_pr.png}
\includeimage
    {demo2_2.png} % Имя файла без расширения (файл должен быть расположен в директории inc/img/)
    {f} % Обтекание (без обтекания)
    {H} % Положение рисунка (см. figure из пакета float)
    {1\textwidth} % Ширина рисунка
    {Состояние стека при пызове подпрограммы или функции} 
    

\includeimage
    {demo2_2_dis.png} % Имя файла без расширения (файл должен быть расположен в директории inc/img/)
    {f} % Обтекание (без обтекания)
    {H} % Положение рисунка (см. figure из пакета float)
    {1\textwidth} % Ширина рисунка
    {Дизассемблер программы}     
    

\includeimage
    {demo2_2_r.png} % Имя файла без расширения (файл должен быть расположен в директории inc/img/)
    {f} % Обтекание (без обтекания)
    {H} % Положение рисунка (см. figure из пакета float)
    {0.5\textwidth} % Ширина рисунка
    {Состояние регистров} 


\includeimage
    {demo2_2_pr.png} % Имя файла без расширения (файл должен быть расположен в директории inc/img/)
    {f} % Обтекание (без обтекания)
    {H} % Положение рисунка (см. figure из пакета float)
    {0.5\textwidth} % Ширина рисунка
    {Схема препываний в протеусе} 

\section{Задание 3}
Подготовим  программу соответствующую заданному алгоритму работы. При инициализации помимо общих директив устанавливаем исходный управляющий код в регистре индикации,  нулевой разряд которого инициирует зажигание светодиода, настраиваем  на вывод порт  микроконтроллера и указатель стека.
В цикле алгоритма на каждой итерации выполняем вывод в порт микроконтроллера управляющего слова, временную задержку, затем циклический сдвиг влево управляющего слова. 
Программные коды основной программы и обработки прерываний дополняем операциями, связанными с обработкой внешних запросов от кнопок, учитывая заданные входы прерываний (INT1 - линия порта PD3, адрес вектора прерываний \$002, маска – бит 7 в регистре GICR; INT2 - линия порта PE0, адрес вектора прерываний \$00D, маска – бит 5 в регистре GICR). 

\includelisting
	{lab2.txt}
	{Код реализации прерывания на плате ATmega8515}

На рисунке \ref{img:lab2.png} продемонстрировано сотояние программы в процессе отладки. На данном рисунке можно отпределить работу стека, изменения состояния регистров и флагов.

\includeimage
    {lab2.png} % Имя файла без расширения (файл должен быть расположен в директории inc/img/)
    {f} % Обтекание (без обтекания)
    {H} % Положение рисунка (см. figure из пакета float)
    {1\textwidth} % Ширина рисунка
    {Состояние программы в процессе отладки} 
    

\includeimage
    {lab2_pr.png} % Имя файла без расширения (файл должен быть расположен в директории inc/img/)
    {f} % Обтекание (без обтекания)
    {H} % Положение рисунка (см. figure из пакета float)
    {1\textwidth} % Ширина рисунка
    {Схема препываний в протеусе} % Подпись рисунка

\newpage
\section{Задание 4}
Запустив программу Proteus ISIS, собрать проект, включающий микроконтроллер, 2 кнопки и 8 светодиодов.
Объединим два запроса прерываний от кнопок с помощью диодной сборки для передачи на вход прерывания микроконтроллера INT2. Обработка прерывания в этом случае начинается с программной идентификации источника запроса прерывания.

\includelisting
	{lab2_2.txt}
	{Код реализации прерывания на плате ATmega8515}

\includeimage
    {lab2_2_pr.png} % Имя файла без расширения (файл должен быть расположен в директории inc/img/)
    {f} % Обтекание (без обтекания)
    {H} % Положение рисунка (см. figure из пакета float)
    {0.75\textwidth} % Ширина рисунка
    {Схема препываний в протеусе} % Подпись рисунка

\chapter{Вывод:}
В результате выполнения лабораторной работы были изучены
системы прерываний микроконтроллеров AVR, а именно прерывания INT0,
INT1 и INT2, освоены системы команд микроконтроллеров AVR, был
получен опыт работы со стеком при вызове подпрограмм и обработчиков
прерываний, а именно понимание работы стека при записи адресов в
пошаговом режиме, и с программированием внешних прерываний. Также,
были получены навыки работы со средой моделирования ISIS Proteus, сборка
схем для моделирования и построение осциллограммы

\end{document}

