 \documentclass{bmstu}
\usepackage{xparse}

\begin{document}

\renewcommand\thefigure{\arabic{figure}} 

\def\numberWork{1}

\makereporttitle
    {Информатика и системы управления} % Название факультета
    {Компьютерные системы и сети} % Название кафедры
    {09.03.01 ПРИКЛАДНАЯ ИНФОРМАТИКА} % направение
    {лабораторной работе №~1 } % Название работы (в дат. падеже)
    { Программирование портов ввода-вывода
микроконтроллеров AVR} % Тема работы
    { Микропроцессорные системы} % Название курса (необязательный аргумент)
    { 19 } % Номер варианта (необязательный аргумент)
    {ИУ6-62Б} % Номер группы
    {
    	{ИУ6-62Б}
    	{А.Е. Медведев} % ФИО студента
    	{Б.И. Бычков} % ФИО преподавателя
    } 


\chapter{Описание выполнения лабораторной работы}

\section{Цель работы:}
\begin{description}
\addtolength{\itemindent}{0.80cm}

\itemsep0em 
\item - изучение системы команд микроконтроллеров AVR и приемов программирования на языке AVR Aссемблер,
\item - получение навыков отладки программ в среде отладки AVR Studio 4 и Proteus,
\item - работа со стартовым набором (платой) STK500.
\end{description} 


\section{Ход работы:}
\textbf{Задание 1.}

	Проверить работу вышеприведенной программы в шаговом режиме
работы с помощью симулятора AVR Studio 4.
Проверяем работу кода, заданного в условии и после изменяем его,
чтобы между переключением светодиодов происходила задержка в 0.5 секунд.
Измененный код программы представлен ниже:

\includelisting
    {demo.asm} % Имя файла с расширением (файл должен быть расположен в директории inc/lst/)
    {Исходный код программы первой лабораторной работы} % Подпись листинга

На рисунках \ref{img:demo1starttime} -- \ref{img:demo1endtime} показано количество пройденного времени до
вхождения в цикл переключения и после его первого прохода.
%\newpage


\includeimage
    {demo1starttime} % Имя файла без расширения (файл должен быть расположен в директории inc/img/)
    {f} % Обтекание (без обтекания)
    {h} % Положение рисунка (см. figure из пакета float)
    {0.5\textwidth} % Ширина рисунка
    {Время до начала цикла} % Подпись рисунка

\includeimage
    {demo1endtime} % Имя файла без расширения (файл должен быть расположен в директории inc/img/)
    {f} % Обтекание (без обтекания)
    {h} % Положение рисунка (см. figure из пакета float)
    {0.5\textwidth} % Ширина рисунка
    {Время после работы цикла} % Подпись рисунка

Схема алгоритма кода, написанного ранее, показана на рисунке \ref{img:demo1schemeofalgorithm}

\includeimage
    {demo1schemeofalgorithm} % Имя файла без расширения (файл должен быть расположен в директории inc/img/)
    {f} % Обтекание (без обтекания)
    {h} % Положение рисунка (см. figure из пакета float)
    {0.5\textwidth} % Ширина рисунка
    {Схема алгоритма} % Подпись рисунка
    
\newpage

\textbf{Задание 2.}
\newline
Проверить работу программы в среде Proteus.
Код, написанный для проверки работы кнопок на осциллографе,
представлен ниже:

\includeimage
    {demo1scheme} % Имя файла без расширения (файл должен быть расположен в директории inc/img/)
    {f} % Обтекание (без обтекания)
    {h} % Положение рисунка (см. figure из пакета float)
    {0.7\textwidth} % Ширина рисунка
    {Схема алгоритма} % Подпись рисунка

\includeimage
    {demo1schemeofanalize} % Имя файла без расширения (файл должен быть расположен в директории inc/img/)
    {f} % Обтекание (без обтекания)
    {h} % Положение рисунка (см. figure из пакета float)
    {0.7\textwidth} % Ширина рисунка
    {Диаграмма цифрового анализатора} % Подпись рисунка

\newpage
\textbf{Задание 3.}
\newline
По заданию преподавателя изменить программу для переключения
светодиодов в заданной последовательности.
Задание для 19-ого варианта:
Непрерывно, перемещая два ВКЛ светодиода,
начиная с 7 разряда, вправо до 0,
и в обратном направлении два ВЫКЛ светодиода
Порт активации PA.
Время переключения – 200 мс.

\includelisting
    {lab1.asm}
    {Код программы по варианту}

На рисунках \ref{img:lab1starttime} -- \ref{img:lab1endtime} показано количество пройденного времени до вхождения в цикл переключения и после его первого прохода.
\includeimage
    {lab1starttime} % Имя файла без расширения (файл должен быть расположен в директории inc/img/)
    {f} % Обтекание (без обтекания)
    {h} % Положение рисунка (см. figure из пакета float)
    {0.5\textwidth} % Ширина рисунка
    {Время до начала цикла} % Подпись рисунка

\includeimage
    {lab1endtime} % Имя файла без расширения (файл должен быть расположен в директории inc/img/)
    {f} % Обтекание (без обтекания)
    {h} % Положение рисунка (см. figure из пакета float)
    {0.6\textwidth} % Ширина рисунка
    {Время после работы цикла} % Подпись рисунка
    
На рисунке \ref{img:lab1scheme} представлена схема в программе Protege.
\includeimage
    {lab1scheme} % Имя файла без расширения (файл должен быть расположен в директории inc/img/)
    {f} % Обтекание (без обтекания)
    {H} % Положение рисунка (см. figure из пакета float)
    {0.6\textwidth} % Ширина рисунка
    {Схема алгоритма} % Подпись рисунка
    
На рисунке \ref{img:lab1digitalanalize} представлена диаграмма цифрового анализатора.
\includeimage
    {lab1digitalanalize} % Имя файла без расширения (файл должен быть расположен в директории inc/img/)
    {f} % Обтекание (без обтекания)
    {H} % Положение рисунка (см. figure из пакета float)
    {0.6\textwidth} % Ширина рисунка
    {Диаграмма цифрового анализатора} % Подпись рисунка

\newpage
Схема алгоритма кода, написанного ранее, показана на рисунке \ref{img:lab1schemeofalgorithm}

\includeimage
    {lab1schemeofalgorithm} % Имя файла без расширения (файл должен быть расположен в директории inc/img/)
    {f} % Обтекание (без обтекания)
    {H} % Положение рисунка (см. figure из пакета float)
    {0.6\textwidth} % Ширина рисунка
    {Схема алгоритма} % Подпись рисунка

\section{Вывод}
В ходе лабораторной работы были изучены системы команд микроконтроллеров AVR и
приемы программирования на языке AVR Aссемблер. Получены навыки
отладки программ в среде отладки AVR Studio 4 и Proteus. Изучены приёмы
работы со стартовым набором (платой) STK500.
\newpage

\end{document}
