\documentclass{bmstu}
\usepackage{xparse}
\usepackage{amsmath}
\begin{document}

% Аргументы, помеченные как необязательные, могут быть пустыми. В таком случае соответствующее этому аргументу поле (например, ФИО консультанта) добавлено не будет.

% Титульная страница

% Отчет
\renewcommand\thefigure{\arabic{figure}}
 
\def\numberWork{1}

\makereporttitle
    {Информатика и системы управления} % Название факультета
    {Компьютерные системы и сети} % Название кафедры
    {09.03.01 Информатика и вычислительная техника} % направение
    {Домашняя работе №~1} % Название работы (в дат. падеже)
    {} % Тема работы
    {Вычислительная математика} % Название курса (необязательный аргумент)
    {19} % Номер варианта (необязательный аргумент)
    {ИУ6-62Б} % Номер группы
    {
    	{ИУ6-62Б}
    	{А.Е.Медведев} % ФИО студента
    	{} % ФИО преподавателя
    } 
    

\section*{Цель работы:}
Изучение метода Гаусса численного решения квадратной СЛАУ с
невырожденной матрицей, оценка числа обусловленности матрицы и исследование его
влияния на погрешность приближенного решения. Изучение метода прогонки решения
СЛАУ с трехдиагональной матрицей.

\section*{Задание 1}
\begin{enumerate}
\item[---] реализовать метод Гаусса решения СЛАУ;
\item[---] провести решение двух заданных СЛАУ методом Гаусса, вычислить нормы невязок
полученных приближенных решений, их абсолютные и относительные погрешности
(использовать 1-норму и бесконечную норму);
\item[---] сравнить полученные результаты с результатами, полученными при использовании
встроенной процедуры метода Гаусса;
\item[---] с использованием реализованного метода Гаусса найти $A_1^{-1}$ и $A_2^{-1}$. Проверить
выполнение равентв $A_i*A_i^{-1} = E$;
\item[---] для каждой системы оценить порядок числа обусловленности матрицы системы и
сделать вывод о его влиянии на точность полученного приближенного решения и
отвечающую ему невязку.
\end{enumerate}

\section*{Ход работы}
Метод Гаусса --- классический метод решения системы линейных алгебраических
уравнений (СЛАУ). Это метод последовательного исключения переменных, когда с
помощью элементарных преобразований система уравнений приводится к равносильной
системе треугольного вида (прямой ход), из которой последовательно, начиная с последних
(по номеру), находятся все переменные системы (обратный ход).
Код программы в Octave Online представлен в листинге \ref{lst:code1.txt}:

\newpage
\includelisting
    {code1.txt}
    {Код программы 1}

Вывод программы 1 представлен в листинге \ref{lst:out1.txt}

\includelisting
    {out1.txt}
    {Вывод программы 1}


%========================================
\newpage
\section*{Задание 2}
\begin{enumerate}
\item[---] реализовать метод прогонки;
\item[---] проверить выполнение достаточных условий применимости для системы из своего
варианта;
\item[---] провести численное решение системы из своего варианта методом прогонки найти
норму его невязки;
\item[---] экспериментально проверить устойчивость найденного решения к малым
возмущениям исходных данных, для чего изменить несколько коэффициентов в правой
части на +0,01, найти решение возмущенной системы и сравнить его с исходным.
\end{enumerate}

\section*{Ход работы}
Метод прогонки используется для решения систем линейных уравнений вида Ax=F,
где A --- трёхдиагональная матрица. Представляет собой вариант метода последовательного
исключения неизвестных.
Система уравнений Ax=F равносильна соотношению:
\[A_ix_{i-1} + B_ix_i + C_ix_{i+1} = F_i \]
Метод прогонки основывается на предположении, что искомые неизвестные связаны
рекуррентным соотношением:
\[x_i = \alpha_{i+1}x_{i+1} + \beta_{i+1} , где i = n-1, n-2, ..., 1.\]
Коэффициенты определяются следующими выражениями:

\begin{equation*}
 \begin{cases}
     \alpha_{i+1} = \frac{-C_i}{A_i \alpha_i + B_i} \\
     \beta_{i+1} = \frac{F_i - A_i \alpha_i}{A_i \alpha_i + B_i} 
 \end{cases}
\end{equation*}

Код программы представлен в листинге \ref{lst:code2.txt}.

\includelisting
    {code2.txt}
    {Код программы 2}

Вывод программы 2 представлен в листинге \ref{lst:out1.txt}

\includelisting
    {out2.txt}
    {Вывод программы 2}

\section*{Вывод:}
	В ходе решения домашнего задания была разработана программа в среде Octave Online. В программе сравнили два метода Гаусса -- самописный и встроенный. Был сделан вывод, что чем больше у метрицы число обусловленностей, тем больше его погрешность. Метод прогонки позволил сделать вывод, что матрица устойчива к малым возмущениям.

\end{document}
