\documentclass{bmstu}
\usepackage{xparse}
\begin{document}

\titlespacing*{\chapter}{0pt}{5pt}{0px} 
\makeatletter %%%%% <---- Starting chapter without a pagebreak
\renewcommand\chapter{\par%
\thispagestyle{plain}% \global\@topnum\z@
\@afterindentfalse \secdef\@chapter\@schapter}
\makeatother       
 
\makereporttitle
    {Информатика и системы управления} % Название факультета
    {Компьютерные системы и сети} % Название кафедры
    {09.03.01 Информатика и вычислительная техника} % направение
    {лабораторной работе №~3} % Название работы (в дат. падеже)
    {Случайные сигналы и их характеристики. Псевдослучайные сигналы} % Тема работы
    {Основы теории цифровой обработки сигналов} % Название курса (необязательный аргумент)
    {19} % Номер варианта (необязательный аргумент)
    {ИУ6-62Б} % Номер группы
    {
    	{ИУ6-62Б}
    	{А.Е.Медведев} % ФИО студента
    	{А.А.Сотников} % ФИО преподавателя
    } 
    
\chapter{Цель работы}
Приобретение практических навыков имитационного моделирования различных видов случайных и псевдослучайных сигналов. Практическое изучение
основных характеристик случайных сигналов, в том числе: плотность вероятности
и функция спектральной плотности мощности.

\chapter{Ход работы}

\begin{enumerate}

\item Смоделировать гистограмму нормального распределения случайной величины
\includelisting
	{lab3_1.txt}
	{Код программы для пункта 1}
\includeimage
	{graph3_1.png}
	{f}
	{H}
	{0.7\textwidth}
	{График гистограммы нормального распределения}

\item Смоделировать случайную величину с нормальным распределением
\includelisting
	{lab3_2.txt}
	{Код программы для пункта 2}
\includeimage
	{graph3_2.png}
	{f}
	{H}
	{0.7\textwidth}
	{График случайной величины с нормальным распределением}
	

\item Смоделировать гистограмму распределения амплитуды сигнала
\includelisting
	{lab3_3.txt}
	{Код программы для пункта 3}
\includeimage
	{graph3_3.png}
	{f}
	{H}
	{0.7\textwidth}
	{Гистограмма распределения амплитуды сигнала}
	
	
\item Смоделировать амплитуду сигнала
\includelisting
	{lab3_4.txt}
	{Код программы для пункта 4}
\includeimage
	{graph3_4.png}
	{f}
	{H}
	{0.7\textwidth}
	{График амплитуды сигнала}
	
	
\item Смоделировать псевдослучайный гармонический сигнал
\includelisting
	{lab3_5.txt}
	{Код программы для пункта 5}
\includeimage
	{graph3_5.png}
	{f}
	{H}
	{0.7\textwidth}
	{График псевдослучайного гармонического сигнала}
	
	
\item Смоделировать пектральную плотность мощности сигнала
\includelisting
	{lab3_6.txt}
	{Код программы для пункта 6}
\includeimage
	{graph3_6.png}
	{f}
	{H}
	{0.7\textwidth}
	{График спектральной плотности мощности сигнала}	

\end{enumerate}

\chapter{Вывод:} 
В ходе выполнения лабораторной работы были смоделированы случайные и псевдо-случайные сигналы. 
Также построен график спектральной плотности мощьности случайного сигнала.

\end{document}







\begin{comment}
На~рисунке~\ref{img:tux} символ семейства Unix-подобных операционных систем Linux.
Он отличается от~<<обычных>> пингвинов желтым цветом клюва и~лап.

\includeimage
    {tux} % Имя файла без расширения (файл должен быть расположен в директории inc/img/)
    {f} % Обтекание (без обтекания)
    {h} % Положение рисунка (см. figure из пакета float)
    {0.25\textwidth} % Ширина рисунка
    {Символ Linux (Tux)} % Подпись рисунка

На~листингах представлен исходный код программы Hello World на~языке программирования C в~двух вариантах оформления.

\includelisting
    {main.c} % Имя файла с расширением (файл должен быть расположен в директории inc/lst/)
    {Исходный код программы Hello World} % Подпись листинга

\includelistingpretty
    {main.c} % Имя файла с расширением (файл должен быть расположен в директории inc/lst/)
    {c} % Язык программирования (необязательный аргумент)
    {Исходный код программы Hello World} % Подпись листинга



\end{comment}
