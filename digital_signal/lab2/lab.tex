\documentclass{bmstu}
\usepackage{xparse}
\begin{document}

% Аргументы, помеченные как необязательные, могут быть пустыми. В таком случае соответствующее этому аргументу поле (например, ФИО консультанта) добавлено не будет.

% Титульная страница

% Отчет

\makereporttitle
    {Информатика и системы управления} % Название факультета
    {Компьютерные системы и сети} % Название кафедры
    {09.03.01 Информатика и вычислительная техника} % направение
    {лабораторной работе №~2} % Название работы (в дат. падеже)
    {Детерминированные сигналы и их основные характеристики} % Тема работы
    {Основы теории обработки цифровых сигналов} % Название курса (необязательный аргумент)
    {} % Номер варианта (необязательный аргумент)
    {ИУ6-62Б} % Номер группы
    {
    	{ИУ6-62Б}
    	{А.Е.Медведев} % ФИО студента
    	{А.А. Сотников} % ФИО преподавателя
    } 
    
\section*{Цель работы}
Приобретение практических навыков имитационного моделирования раз-
личных видов детерминированных сигналов. Экспериментальное изучение ос-
новных характеристик дискретных сигналов, в том числе: энергия и средняя
мощность сигнала на интервале, амплитудный и энергетический спектры, спектр
мощности и функция спектральной плотности мощности.

\section*{Задачи}
\begin{enumerate}
\item \text{Провести имитационное моделирование детерминированного сигнала в виде одиночного импульса заданного типа длительностью $\tau$ и периодической последовательности из N подобных импульсов с периодом T.}
\item Провести имитационное моделирование гармонического сигнала с заданным
типом модуляции длительностью $\tau$ с девиацией частоты в диапазоне $f \in
[f_{min} ; f_{max} ]$ на временном интервале $t \in [t_{min} ; t_{max} ]$. 
Частота дискретизации должна быть выбрана в соответствии с требованиями теоремы Найквиста-Котельникова, а именно $f_d > 2*f_{max}$ , где $_{max}$ ---
максимальная частота в спектре моделируемого сигнала.
\item Экспериментально рассчитать энергию моделируемого гармонического сигнала во временной и частотной областях и подтвердить выполнение теоремы Парсеваля.
\item Оценить среднюю мощность моделируемого гармонического сигнала на заданном интервале.
\item Построить амплитудный и энергетический спектры, спектр мощности и
функцию спектральной плотности мощности гармонического сигнала. Так
как спектральные характеристики действительного сигнала симметричны
относительно нулевой частоты, то рекомендуется при выполнении работы
на графиках изображать только положительные частоты, удваивая значе-
ния характеристики в области положительных частот за счет скрытых на
графике соответствующих отрицательных частот. Такой вид изображения
является более интуитивно понятным, так как в физических процессах
отрицательные частоты отсутствуют.

\end{enumerate}

\section*{Выполнение}

\includelisting
	{code2.py}
	{Код программы}

\includeimage
    {Figure_1.png} % Имя файла без расширения (файл должен быть расположен в директории inc/img/)
    {f} % Обтекание (без обтекания)
    {H} % Положение рисунка (см. figure из пакета float)
    {0.8\textwidth} % Ширина рисунка
    {Моделирование треугольного импульса} % Подпись рисунка

\includeimage
    {Figure_2.png} % Имя файла без расширения (файл должен быть расположен в директории inc/img/)
    {f} % Обтекание (без обтекания)
    {H} % Положение рисунка (см. figure из пакета float)
    {0.8\textwidth} % Ширина рисунка
    {Моделирование последовательности треугольных импульсов} % Подпись рисунка

\includeimage
    {Figure_3.png} % Имя файла без расширения (файл должен быть расположен в директории inc/img/)
    {f} % Обтекание (без обтекания)
    {H} % Положение рисунка (см. figure из пакета float)
    {0.8\textwidth} % Ширина рисунка
    {Моделирование сигнала с линейной частотной модуляцией} % Подпись рисунка

\includeimage
    {Figure_4.png} % Имя файла без расширения (файл должен быть расположен в директории inc/img/)
    {f} % Обтекание (без обтекания)
    {H} % Положение рисунка (см. figure из пакета float)
    {0.8\textwidth} % Ширина рисунка
    {Моделирование сигнала во временной области} % Подпись рисунка

\includeimage
    {Figure_5.png} % Имя файла без расширения (файл должен быть расположен в директории inc/img/)
    {f} % Обтекание (без обтекания)
    {H} % Положение рисунка (см. figure из пакета float)
    {0.8\textwidth} % Ширина рисунка
    {Расчет амплитудного спектра сигнала} % Подпись рисунка

\includeimage
    {Figure_6.png} % Имя файла без расширения (файл должен быть расположен в директории inc/img/)
    {f} % Обтекание (без обтекания)
    {H} % Положение рисунка (см. figure из пакета float)
    {1\textwidth} % Ширина рисунка
    {Расчет энергетического спектра сигнала} % Подпись рисунка

\includeimage
    {Figure_7.png} % Имя файла без расширения (файл должен быть расположен в директории inc/img/)
    {f} % Обтекание (без обтекания)
    {H} % Положение рисунка (см. figure из пакета float)
    {0.8\textwidth} % Ширина рисунка
    {Расчет спектра мощности сигнала} % Подпись рисунка

\includeimage
    {Figure_8.png} % Имя файла без расширения (файл должен быть расположен в директории inc/img/)
    {f} % Обтекание (без обтекания)
    {H} % Положение рисунка (см. figure из пакета float)
    {0.8\textwidth} % Ширина рисунка
    {Функция спектральной мощности сигнала} % Подпись рисунка

\includeimage
    {Figure_9.png} % Имя файла без расширения (файл должен быть расположен в директории inc/img/)
    {f} % Обтекание (без обтекания)
    {H} % Положение рисунка (см. figure из пакета float)
    {0.8\textwidth} % Ширина рисунка
    {Функция спектральной плотности мощности сигнала} % Подпись рисунка

\section*{Вывод}
В ходе работы лабораторной работы были промоделированы различные виды сигналов. Были расчитаны амплитудные и энергетические спектры сигналов и спектор можности сигналов.

\end{document}
\begin{comment}
На~рисунке~\ref{img:tux} символ семейства Unix-подобных операционных систем Linux.
Он отличается от~<<обычных>> пингвинов желтым цветом клюва и~лап.

\includeimage
    {tux} % Имя файла без расширения (файл должен быть расположен в директории inc/img/)
    {f} % Обтекание (без обтекания)
    {h} % Положение рисунка (см. figure из пакета float)
    {0.25\textwidth} % Ширина рисунка
    {Символ Linux (Tux)} % Подпись рисунка

На~листингах представлен исходный код программы Hello World на~языке программирования C в~двух вариантах оформления.

\includelisting
    {main.c} % Имя файла с расширением (файл должен быть расположен в директории inc/lst/)
    {Исходный код программы Hello World} % Подпись листинга

\includelistingpretty
    {main.c} % Имя файла с расширением (файл должен быть расположен в директории inc/lst/)
    {c} % Язык программирования (необязательный аргумент)
    {Исходный код программы Hello World} % Подпись листинга



\end{comment}
