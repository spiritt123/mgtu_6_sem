\documentclass{bmstu}
\usepackage{xparse}
\begin{document}


\titlespacing*{\chapter}{0pt}{5pt}{0px} 
\makeatletter %%%%% <---- Starting chapter without a pagebreak
\renewcommand\chapter{\par%
\thispagestyle{plain}% \global\@topnum\z@
\@afterindentfalse \secdef\@chapter\@schapter}
\makeatother  

\makereporttitle
    {Информатика и системы управления} % Название факультета
    {Компьютерные системы и сети} % Название кафедры
    {09.03.01 Информатика и вычислительная техника} % направение
    {лабораторной работе №~4} % Название работы (в дат. падеже)
    {Простые и сложные сигналы.} % Тема работы
    {Основы теории цифровой обработки сигналов} % Название курса (необязательный аргумент)
    {19} % Номер варианта (необязательный аргумент)
    {ИУ6-62Б} % Номер группы
    {
    	{ИУ6-62Б}
    	{А.Е.Медведев} % ФИО студента
    	{А.А.Сотников} % ФИО преподавателя
    } 
    
\chapter{Цель работы:}
Приобретение практических навыков, освоение программных средств
имитационного моделирования простых и сложных сигналов с различными
характеристиками, применяемых в радиолокации, гидролокации и других
отраслях науки и техники. Практическое изучение способов увеличения
разрешающей способности измерения смещения сигналов по времени и
частоте.

\chapter{Ход работы}

\begin{enumerate}

\item Смоделировать простейшие графики автокоррекции
\includelisting
        {lab4_1.m}
        {Код программы для пункта 1}
\includeimage
        {graph4_1.png}
        {f}
        {H}
        {0.7\textwidth}
        {График автокорреляции тонального импульса}

\includeimage
        {graph4_2.png}
        {f}
        {H}
        {0.7\textwidth}
        {Проекция графика автокорреляции}

\item Смоделировать работу функции неопределённость тонального импульса
\includelisting
        {lab4_2.m}
        {Код программы для пункта 2}
\includeimage
        {graph4_3.png}
        {f}
        {H}
        {0.7\textwidth}
        {График функции неопределенности тонального импульса}

\includeimage
        {graph4_4.png}
        {f}
        {H}
        {0.7\textwidth}
        {Проекция графика функции неопределенности}

\item Смоделировать автокоррекцию по времени и частоте
\includelisting
        {lab4_3.m}
        {Код программы для пункта 3}
\includeimage
        {graph4_5.png}
        {f}
        {H}
        {0.7\textwidth}
        {График автокорреляционной функции по частоте}
\includeimage
        {graph4_6.png}
        {f}
        {H}
        {0.7\textwidth}
        {График автокорреляционной функции по времени}

\includeimage
        {graph4_7.png}
        {f}
        {H}
        {0.7\textwidth}
        {График автокорреляционной функции для ЛЧМ-импульса}
\item Смоделировать графики
\includelisting
        {lab4_4.m}
        {Код программы для пункта 3}
\includeimage
        {graph4_8.png}
        {f}
        {H}
        {0.7\textwidth}
        {График двумерной автокорреляционной функции для ЛЧМ-импульса}
\includeimage
        {graph4_9.png}
        {f}
        {H}
        {0.7\textwidth}
        {Проекция двумерной автокорреляционной функции для ЛЧМ-импульса}

\item Смоделировать работы функции неопределённости для ЛЧМ и её автокорреляционной функции.
\includelisting
        {lab4_5.m}
        {Код программы для пункта 3}
\includeimage
        {graph4_10.png}
        {f}
        {H}
        {0.7\textwidth}
        {График функции неопределенности для ЛЧМ-импульса}
\includeimage
        {graph4_11.png}
        {f}
        {H}
        {0.7\textwidth}
        {Проекция функции неопределенности для ЛЧМ-импульса}
\includeimage
        {graph4_12.png}
        {f}
        {H}
        {0.7\textwidth}
        {График автокорреляционной функции по частоте для ЛЧМ-импульса}
\includeimage
        {graph4_13.png}
        {f}
        {H}
        {0.7\textwidth}
        {График автокорреляционной функции по времени для ЛЧМ-импульса}

\end{enumerate}

\chapter{Вывод:}
В ходе лабораторной работы были изучены способы моделерования простых и сложных сигналов, функции автокоррекции и способы увеличения разрешающей способности.

\end{document}
\begin{comment}
На~рисунке~\ref{img:tux} символ семейства Unix-подобных операционных систем Linux.
Он отличается от~<<обычных>> пингвинов желтым цветом клюва и~лап.

\includeimage
    {tux} % Имя файла без расширения (файл должен быть расположен в директории inc/img/)
    {f} % Обтекание (без обтекания)
    {h} % Положение рисунка (см. figure из пакета float)
    {0.25\textwidth} % Ширина рисунка
    {Символ Linux (Tux)} % Подпись рисунка

На~листингах представлен исходный код программы Hello World на~языке программирования C в~двух вариантах оформления.

\includelisting
    {main.c} % Имя файла с расширением (файл должен быть расположен в директории inc/lst/)
    {Исходный код программы Hello World} % Подпись листинга

\includelistingpretty
    {main.c} % Имя файла с расширением (файл должен быть расположен в директории inc/lst/)
    {c} % Язык программирования (необязательный аргумент)
    {Исходный код программы Hello World} % Подпись листинга



\end{comment}
