\documentclass{bmstu}
\usepackage{xparse}
\begin{document}

\titlespacing*{\chapter}{0pt}{5pt}{0px} 
\makeatletter %%%%% <---- Starting chapter without a pagebreak
\renewcommand\chapter{\par%
\thispagestyle{plain}% \global\@topnum\z@
\@afterindentfalse \secdef\@chapter\@schapter}
\makeatother  

\makereporttitle
    {Информатика и системы управления} % Название факультета
    {Компьютерные системы и сети} % Название кафедры
    {09.03.01 Информатика и вычислительная техника} % направение
    {лабораторной работе №~7} % Название работы (в дат. падеже)
    {Имитационное моделирование двумерных сигналов на примере изображений} % Тема работы
    {Основы теории цифровой обработки сигналов} % Название курса (необязательный аргумент)
    {} % Номер варианта (необязательный аргумент)
    {ИУ6-62Б} % Номер группы
    {
        {ИУ6-62Б}
        {А.Е.Медведев} % ФИО студента
        {А.А.Сотников} % ФИО преподавателя
    } 
    
\chapter{Цель работы:}
   Приобретение практических навыков, освоение программных средств ими-
тационного моделирования комплексных и многомерных сигналов. Практическое
изучение программных средств формирования и воспроизведения изображений и
видеофрагментов.
    
\chapter{Ход работы}

\begin{enumerate}

\item Смоделировать простейшие графики автокоррекции
\includelisting
        {lab7_1.m}
        {Код программы}
\includeimage
        {graph7_1.png}
        {f}
        {H}
        {0.5\textwidth}
        {Графическое изображение функции f(z) = z}
\includeimage
        {graph7_2.png}
        {f}
        {H}
        {0.5\textwidth}
        {Гафическое изображение функции $f(z) = ze^z$}  
\includeimage
        {graph7_3.png}
        {f}
        {H}
        {0.5\textwidth}
        {Графическое изображение функции $f(z) = \frac{z^2-i}{z^3+i}$}
\includeimage
        {graph7_4.png}
        {f}
        {H}
        {0.5\textwidth}
        {Добавление нормального шума}
\includeimage
        {graph7_5.png}
        {f}
        {H}
        {0.5\textwidth}
        {Объединение изображений}
       
\end{enumerate}

\chapter{Вывод:} 
В ходе выполнения лабораторной работы были
приобретены практические навыки, освоены программные средства
имитационного моделирования комплексных и многомерных сигналов.
Изучены программные средства формирования и воспроизведения
изображений и видеофрагментов.

\end{document}
\begin{comment}
На~рисунке~\ref{img:tux} символ семейства Unix-подобных операционных систем Linux.
Он отличается от~<<обычных>> пингвинов желтым цветом клюва и~лап.

\includeimage
    {tux} % Имя файла без расширения (файл должен быть расположен в директории inc/img/)
    {f} % Обтекание (без обтекания)
    {h} % Положение рисунка (см. figure из пакета float)
    {0.25\textwidth} % Ширина рисунка
    {Символ Linux (Tux)} % Подпись рисунка

На~листингах представлен исходный код программы Hello World на~языке программирования C в~двух вариантах оформления.

\includelisting
    {main.c} % Имя файла с расширением (файл должен быть расположен в директории inc/lst/)
    {Исходный код программы Hello World} % Подпись листинга

\includelistingpretty
    {main.c} % Имя файла с расширением (файл должен быть расположен в директории inc/lst/)
    {c} % Язык программирования (необязательный аргумент)
    {Исходный код программы Hello World} % Подпись листинга



\end{comment}
