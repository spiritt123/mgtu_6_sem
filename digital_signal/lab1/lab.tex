\documentclass{bmstu}
\usepackage{xparse}
\usepackage{mathtools}
\begin{document}

\makereporttitle
    {Информатика и системы управления} % Название факультета
    {Компьютерные системы и сети} % Название кафедры
    {09.03.01 Информатика и вычислительная техника} % направение
    {лабораторной работе №~1} % Название работы (в дат. падеже)
    {Непрерывные, дискретные и цифровые сигналы} % Тема работы
    {Основы теории обработки цифровых сигналов} % Название курса (необязательный аргумент)
    {} % Номер варианта (необязательный аргумент)
    {ИУ6-62Б} % Номер группы
    {
    	{ИУ6-62Б}
    	{А.Е.Медведев} % ФИО студента
    	{А.А.Сотников} % ФИО преподавателя
    } 
    

\section*{Цель работы}
Практическое исследование этапов аналого-цифрового преобразования сиг-
налов с использованием современных средств имитационного моделирования.
Сравнительный анализ аналогового, дискретного и цифрового сигналов. Приоб-
ретение практических навыков применения программных средств имитационного
моделирования цифровых сигналов.

\section*{Задачи}
\begin{enumerate} 
\item Выполнить имитационное моделирование аналогового гармонического сигнала одной частоты, описываемого функцией $x(t) = \exp(-t) * \cos(2\pi t) + 1$ на временном интервале $t \in [t_{min} ; t_{max} ]$ с использованием символьных переменных;
\item Построить график функции, описывающей аналоговый сигнал;
\item Выполнить моделирование аналого-цифрового преобразования с частотой дискретизации fd и разрядностью b. Кодирование сигнала реализовать с помощью прямого, обратного или дополнительного кода;
\item Построить графики соответствующих функций для дискретного, квантованного и цифрового сигналов;
\item Оценить параметры шума квантования сигнала, построить гистограмму статистического распределения абсолютной погрешности квантования и сопоставить полученные результаты с теоретическими значениями
\end{enumerate}

\newpage

\section*{Решение}
Код, представленный в листинге \ref{lst:code.py}, описывает работу аналогово, дискретного, квантованного и цифрового сигналов. \\
Графики представлены на рисунке \ref{img:functions}.

\includelistingpretty
    {code.py} 
    {Python} % Язык программирования (необязательный аргумент)
    {Исходный код программы} % Подпись листинга
    
\includeimage
	{functions}
	{f}
	{H}
	{1\textwidth}
	{Графики различноых сигналов}

\section*{Вывод}
В ходе выполнени лабораторной работы были изучены форматы сигналов, способы перехода от аналогового к дискретному сигналу, уровни квантования и частоты дискритизации.

\end{document}
