\documentclass{bmstu}
\usepackage{xparse}
\begin{document}


\makereporttitle
    {Информатика и системы управления} % Название факультета
    {Компьютерные системы и сети} % Название кафедры
    {09.03.01 Информатика и вычислительная техника} % направение
    {Лабораторная работа №~3} % Название работы (в дат. падеже)
    {Проектирование устройств управления на основе ПЛИС} % Тема работы
    {Основы проектирования устройств ЭВМ.} % Название курса (необязательный аргумент)
    {38} % Номер варианта (необязательный аргумент)
    {ИУ6-62Б} % Номер группы
    {
    	{ИУ6-62Б}
    	{А.Е.Медведев} % ФИО студента
    	{} % ФИО преподавателя
    } 
    
\section*{Цель работы:}
Закрепление на практике теоретических знаний о
способах реализации устройств управления, исследование способов
организации узлов ЭВМ, освоение принципов проектирования цифровых
устройств на основе ПЛИС.

По варианту в данном домашнем задании будет графф под номером 2. Название платы --- Nexus2
Сам граф приведён на рисунке \ref{img:graff.png}

\includeimage
	{graff.png}
	{f}
	{H}
	{0.5\textwidth}
	{Граф задания}


\begin{table}[ht]
        \begin{center}
        \caption{Варианты диаграмм и активных сигналов.}
        \begin{tabular}{|c|c|c|c|c|c|}
                \hline
                S1 & S2 & S3 & S4 & S5 & S6\\
                \hline
                1 & 3,5 & 6 & - & 7 & 2,4\\
                \hline
        \end{tabular}
        \end{center}
\end{table}
                

\begin{table}[ht]
        \begin{center}
        \caption{Условия переходов и наименование отладочной платы}
        \begin{tabular}{|c|c|c|c|c|c|c|c|c|c|c|c|c|c|c|c|}
                \hline 
                У1 & У2 & У3 & У4 & У5 & У6 & У7 & У8 & У9 & У10 & У11 & У12 & У13 & У14 & У15 \\
                \hline
                @ & CE & AC & @ & !E+C & F & @ & !DF & B & @ & !A!C & A!B & @ & @ & ABCF\\
                \hline
        \end{tabular}
        \end{center}
\end{table}

\begin{table}[ht]
        \begin{center}
        \caption{Условия переходов и наименование отладочной платы}
        \begin{tabular}{|c|c|c|c|c|c|c|c|c|c|c|c|c|c|c|c|}
                \hline 
                У1 & У2 & У3 & У4 & У5 & У6 & У7 & У8 & У9 & У10 & У11 & У12 & У13 & У14 & У15 \\
                \hline
                - & - & - & - & - & - & - & - & - & - & - & - & - & - & - \\
                \hline
        \end{tabular}
        \end{center}
\end{table}

\section*{Этап 1}

В лабораторной работе необходимо разработать и реализовать на ПЛИС
XC3S200 или XC3E-500 управляющий автомат схемного типа,
обрабатывающий входное командное слово С={A,B,C,D,E,F}, выдающий
сигналы управления M={M0,...,Mk-1} операционному блоку
Конечный граф представлен на рисунке \ref{img:graff_fix.png}

\includeimage
	{graff_fix.png}
	{f}
	{H}
	{0.7\textwidth}
	{Исправленные граф}

\section*{Этап 2}
	
Результаты моделирования приведены на рисунках \ref{img:photo.png}. 

\includeimage
	{photo.png}
	{f}
	{H}
	{0.9\textwidth}
	{Переключения состояний программы}

Исходный код модуля верхнего уровня разрабатываемого устройства
приведен в листинге \ref{lst:modul_hide_level.vhd}.

\includelisting
	{modul_hide_level.vhd}
	{Описание устройства}

\section*{Вывод:}
В ходе выполнения лабораторной работы были закреплены на
практике навыки разработки устройств управления на языке VHDL (в данном
случае – устройства управления с жесткой логикой на основе цифровых
автоматов)

\end{document}
\begin{comment}
На~рисунке~\ref{img:tux} символ семейства Unix-подобных операционных систем Linux.
Он отличается от~<<обычных>> пингвинов желтым цветом клюва и~лап.

\includeimage
    {tux} % Имя файла без расширения (файл должен быть расположен в директории inc/img/)
    {f} % Обтекание (без обтекания)
    {h} % Положение рисунка (см. figure из пакета float)
    {0.25\textwidth} % Ширина рисунка
    {Символ Linux (Tux)} % Подпись рисунка

На~листингах представлен исходный код программы Hello World на~языке программирования C в~двух вариантах оформления.

\includelisting
    {main.c} % Имя файла с расширением (файл должен быть расположен в директории inc/lst/)
    {Исходный код программы Hello World} % Подпись листинга

\includelistingpretty
    {main.c} % Имя файла с расширением (файл должен быть расположен в директории inc/lst/)
    {c} % Язык программирования (необязательный аргумент)
    {Исходный код программы Hello World} % Подпись листинга



\end{comment}
