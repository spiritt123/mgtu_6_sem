\documentclass{bmstu}
\usepackage{xparse}
\begin{document}

% Аргументы, помеченные как необязательные, могут быть пустыми. В таком случае соответствующее этому аргументу поле (например, ФИО консультанта) добавлено не будет.

% Титульная страница

% Отчет

\makereporttitle
    {Информатика и системы управления} % Название факультета
    {Компьютерные системы и сети} % Название кафедры
    {09.03.01 ПРИКЛАДНАЯ ИНФОРМАТИКА} % направение
    {лабораторной работе №~1} % Название работы (в дат. падеже)
    {Проектирование систем на кристалле на основе ПЛИС} % Тема работы
    {Основы проектирования устройств ЭВМ} % Название курса (необязательный аргумент)
    {} % Номер варианта (необязательный аргумент)
    {ИУ6-62Б} % Номер группы
    {
    	{ИУ6-62Б}
    	{А.Е. Медведев} % ФИО студента
    	{С.В. Ибрагимов} % ФИО преподавателя
    } 

\section*{Цель работы}
Изучение основ построения микропроцессорных систем на ПЛИС. В ходе работы студенты
ознакомятся с принципами построения систем на кристалле (СНК) на основе ПЛИС, получат
навыки проектирования СНК в САПР Altera Quartus II, выполнят проектирование и
верификацию системы с использованием отладочного комплекта Altera DE1Board

\section*{Задачи}
\begin{enumerate}
\item Создать новый проект
\item Создать новый модуль системы на кристалле QSYS.
\item \text{Добавить модуль $ c: \backslash user \backslash sopc01 \backslash nios.qsys$ в проект sopc01}
\item Назначить модуль nios.qsys в качестве модуля верхнего уровня
\item Выполнить синтез проекта.
\item Назначить портам проекта контакты микросхемы
\item Выполнить синтез проекта. 
\item Создать программный проект Nios2.
\item Выполнить прошивку проекта в ПЛИС.
\item Выполнить верификацию проекта с использованием программы терминала.
\item Изменить параметр System ID на 32-х разрядный код, состоящий из номера группы и варианта.
\item Доработать код программного проекта: добавить строки, передающие по UART значение SystemID в виде четырех байт символов в ASCII формате.

\end{enumerate}

\section*{Выполнение}
Порядок выполнения пунктов представлен на рисунках \ref{img:1.png} --- \ref{img:6.png}, код программы представлен в листинге \ref{lst:code.c}.

\includeimage
    {1.png} % Имя файла без расширения (файл должен быть расположен в директории inc/img/)
    {f} % Обтекание (без обтекания)
    {H} % Положение рисунка (см. figure из пакета float)
    {0.8\textwidth} % Ширина рисунка
    {Окно модуля Qsys после назначения базовых адресов} % Подпись рисунка

\includeimage
    {2.png} % Имя файла без расширения (файл должен быть расположен в директории inc/img/)
    {f} % Обтекание (без обтекания)
    {H} % Положение рисунка (см. figure из пакета float)
    {0.8\textwidth} % Ширина рисунка
    {Модуль Pin Planner.} % Подпись рисунка

\includeimage
    {3.png} % Имя файла без расширения (файл должен быть расположен в директории inc/img/)
    {f} % Обтекание (без обтекания)
    {H} % Положение рисунка (см. figure из пакета float)
    {0.8\textwidth} % Ширина рисунка
    {Назначенные пины} % Подпись рисунка

\includeimage
    {4.png} % Имя файла без расширения (файл должен быть расположен в директории inc/img/)
    {f} % Обтекание (без обтекания)
    {H} % Положение рисунка (см. figure из пакета float)
    {0.8\textwidth} % Ширина рисунка
    {Результат назначения пинов} % Подпись рисунка

\includeimage
    {5.png} % Имя файла без расширения (файл должен быть расположен в директории inc/img/)
    {f} % Обтекание (без обтекания)
    {H} % Положение рисунка (см. figure из пакета float)
    {0.8\textwidth} % Ширина рисунка
    {Окно модуля программирования ПЛИС} % Подпись рисунка

\includeimage
    {6.png} % Имя файла без расширения (файл должен быть расположен в директории inc/img/)
    {f} % Обтекание (без обтекания)
    {H} % Положение рисунка (см. figure из пакета float)
    {0.8\textwidth} % Ширина рисунка
    {Вывод программы} % Подпись рисунка
    
\newpage
\includelistingpretty
    {code.c} % Имя файла с расширением (файл должен быть расположен в директории inc/lst/)
    {c} % Язык программирования (необязательный аргумент)
    {Исходный код программы} % Подпись листинга


\section*{Вывод}
В ходе лабораторной работы был создан проект в среде Quartus II. Для кристала Altera Quartus II были выбраны нужные связи и установлены входы и выходы. Написана прошивка и загружена на учебную плату. Проверена корректность написанной программы.

\end{document}
\begin{comment}
На~рисунке~\ref{img:tux} символ семейства Unix-подобных операционных систем Linux.
Он отличается от~<<обычных>> пингвинов желтым цветом клюва и~лап.

\includeimage
    {tux} % Имя файла без расширения (файл должен быть расположен в директории inc/img/)
    {f} % Обтекание (без обтекания)
    {h} % Положение рисунка (см. figure из пакета float)
    {0.25\textwidth} % Ширина рисунка
    {Символ Linux (Tux)} % Подпись рисунка

На~листингах представлен исходный код программы Hello World на~языке программирования C в~двух вариантах оформления.

\includelisting
    {main.c} % Имя файла с расширением (файл должен быть расположен в директории inc/lst/)
    {Исходный код программы Hello World} % Подпись листинга

\includelistingpretty
    {main.c} % Имя файла с расширением (файл должен быть расположен в директории inc/lst/)
    {c} % Язык программирования (необязательный аргумент)
    {Исходный код программы Hello World} % Подпись листинга



\end{comment}
