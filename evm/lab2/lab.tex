\documentclass{bmstu}
\usepackage{xparse}
\usepackage{diagbox}
\usepackage{karnaugh-map}
\begin{document}

\newcolumntype{M}[1]{>{\centering\arraybackslash}m{#1}}

\makereporttitle
    {Информатика и системы управления} % Название факультета
    {Компьютерные системы и сети} % Название кафедры
    {09.03.01 Информатика и вычислительная техника} % направение
    {лабораторной работе №~2} % Название работы (в дат. падеже)
    {Проектирование цифровых устройств на основе ПЛИС} % Тема работы
    {Основы проектирования устройств ЭВМ} % Название курса (необязательный аргумент)
    {19} % Номер варианта (необязательный аргумент)
    {ИУ6-62Б} % Номер группы
    {
    	{ИУ6-62Б}
    	{А.Е.Медведев} % ФИО студента
    	{С.В. Ибрагимов} % ФИО преподавателя
    } 

\section*{Цель работы}
Закрепление на практике теоретических сведений, полученных при
изучении методики проектирования цифровых устройств на основе программируемых
логических интегральных схем (ПЛИС), получение необходимых навыков работы с системой
автоматизированного проектирования ISE WebPack устройств на основе ПЛИС фирмы
Xilinx, изучение аппаратных и программных средств моделирования, макетирования и
отладки устройств на основе ПЛИС.

\section*{Данные по варианту}

\begin{table}[ht]
	\begin{center}
	\caption{Таблица данных по вариантов}
	\begin{tabular}{|c|c|c|c|c|}
		\hline
		Набор   & State0 & State1 & State2 & State3\\
		\hline
		XC3S200 & 11     & 00     & 01     & 10\\
		\hline
	\end{tabular}
	\end{center}
\end{table}

\section*{Задание 1. }
Выполнить кодирование состояний автомата в соответствии с индивидуальным вариантом.
Функциональная схема разрабатываемого устройства показана на рисунке \ref{img:image1.png}.

\includeimage
	{image1.png}
	{f}
	{H}
	{0.5\textwidth}
	{Функциональная схема разрабатываемого устройства}

Диаграмма состояний автомата подавления дребезга представлена на рисунке \ref{img:image2.png}.

\includeimage
	{image2.png}
	{f}
	{H}
	{0.9\textwidth}
	{Диаграмма состояний автомата подавления дребезга}

Функциональная схема устройства показана на рисунке \ref{img:image3.png}.

\includeimage
	{image3.png}
	{f}
	{H}
	{0.9\textwidth}
	{Функциональная схема устройства}
	
	
\begin{table}[ht]
	\begin{center}
	\caption{Таблица выходов}
	\begin{tabular}{|c|c|c|c|c|}
		\hline
		Состояние & State0 & State1 & State2 & State3\\
		\hline
		Двоичный код состояния & 11     & 00     & 01     & 10\\
		\hline
		CNT        & 0      & 1      & 1      & 0\\
		\hline
		DLY\_EN    & 0      & 1      & 0      & 1 \\
		\hline
	\end{tabular}
	\end{center}
\end{table}

Найдём функции для сигналов CNT и DLY\_EN. \\

$	\text{CNT} = \overline{S(1)}\\
	\text{DLY\_EN} = \overline{S(0)}
$


\begin{table}[ht]
	\begin{center}
	\caption{Сигналы SN(*) и D*}
	\begin{tabular}{|c|c|c|c|c|c|c|c|l|}
		\hline
		   \scriptsize{COUNT} & 
		   \scriptsize{DLY\_OVF} & S1(t) & S0(t) & S1(t+1) & S0(t+1) & SN(1) & SN(0) & \scriptsize{Описание события} \\
		\hline
		0     & x        & 1     & 1     & 1       & 1       & 1     & 1     &  \scriptsize{Ожидание нажатия кнопки}\\
		\hline
		1     & x        & 1     & 1     & 0       & 0       & 0     & 0     &  \scriptsize{Нажатие кнопки}\\
		\hline
		x     & 0        & 0     & 0     & 0       & 0       & 0     & 0     &  \scriptsize{Ожидание окончания счета}\\
		\hline
		x     & 1        & 0     & 0     & 0       & 1       & 0     & 1     &  \scriptsize{Конец счета}\\
		\hline
		1     & x        & 0     & 1     & 0       & 1       & 0     & 1     &  \scriptsize{Ожидание отпускания}\\
		\hline
		0     & x        & 0     & 1     & 1       & 0       & 1     & 0     &  \scriptsize{Отпускание кнопки}\\
		\hline
		x     & 0        & 1     & 0     & 1       & 0       & 1     & 0     &  \scriptsize{Ожидание окончания счета}\\
		\hline
		x     & 1        & 1     & 0     & 1       & 1       & 1     & 1     &  \scriptsize{Конец счета}\\
		\hline
	\end{tabular}
	\end{center}
\end{table}

Найдём минимальные булевые функции для сигнала SN через карты карно.

\begin{table}[ht]
\begin{center}
	\caption{Карты карно для SN(1) и SN(0)}
	
	\begin{karnaugh-map}[4][4][1][COUNT / DLY\_OVF][S(1) / S(0)]
        \manualterms{0,0,0,0,
        			 1,1,0,0,
        			 1,1,1,1,
        			 1,1,0,0}
        \implicant{4}{13}
        \implicant{8}{10}
	\end{karnaugh-map}
	\quad
	\begin{karnaugh-map}[4][4][1][COUNT / DLY\_OVF][S(1) / S(0)]
        \manualterms{0,1,0,1,
        			 0,0,1,1,
        			 0,1,0,1,
        			 1,1,0,0}
        \implicant{7}{6}
        \implicant{12}{13}
        \implicantedge{1}{3}{9}{11}
	\end{karnaugh-map}
\end{center}
\end{table}

Тогда $\text{SN(1)} = \overline{\text{COUNT}} ~ \text{S(0)} \vee \text{S(1) S(0)} $,  \\
а  $\text{SN(0)} = \text{DLY\_OVF S(1)} \vee \text{COUNT $\overline{S(1)}$ S(0)} \vee \overline{\text{COUNT}}~ \text{S(1) S(0)}$. 

\section*{Задание 2. }
Разработайте текстовое описание модуля в соответствии с полученными
функциями DLY\_EN, CNT, SN(0), SN(1)\\
Код программы приведен в листинге \ref{lst:lab2_2.vhd}

\includelisting
	{lab2_2.vhd}
	{Код на VHDL}

\section*{Задание 3}
В интегрированном редакторе тестов САПР Xilinx ISE разработать тест для полученного устройства и выполнить моделирование его работы в симуляторе Modelsim.
На рисунке \ref{img:image5.png} показаны входные исходные для теста в графическом представлении.

\includeimage
	{image5.png}
	{f}
	{H}
	{1\textwidth}
	{Исходные данные теста}


	Результаты теста показаны на рисунке \ref{img:image6.png}.
\includeimage
	{image6.png}
	{f}
	{H}
	{1\textwidth}
	{Результаты теста}

	Как видно из результатов теста – устройство работает корректно.

\section*{Задание 4}
	Разработать устройство управления, принимающее 16-разрядное информационное слово Q[0..15] и управляющее их последовательной выдачей по шине D[0..3] на декодер 7-сегментных индикаторов.
	Исходный код модуля представлен в листинге \ref{lst:control.vhd}

\includelisting
	{control.vhd}
	{Настройка управляющего модуля}

На рисунках \ref{img:image7.png} -- \ref{img:image9.png} показаны временные диаграммы, полученные при тестировании данного модуля.

\includeimage
	{image7.png}
	{f}
	{H}
	{1\textwidth}
	{Создание теста}
	
\includeimage
	{image8.png}
	{f}
	{H}
	{1\textwidth}
	{Результаты теста}
	
\includeimage
	{image9.png}
	{f}
	{H}
	{1\textwidth}
	{Результаты теста}
	
\section*{Задние 5}
Разработать поведенческое VHDL описание схемы преобразования четырехразрядного информационного кода D[0..3] в код активизации 7-сегментного индикатора LED[0..7].
Исходный код представлен в листинге \ref{lst:sev_seg.vhd}

\includelisting
	{sev_seg.vhd}
	{Настройка семисегментного модуля}


\section*{Задание 6}
	В редакторе схем САПР ISE добавить описание основного модуля.
	Исходный код представлен в листинге \ref{lst:main.vhd} 

\includelisting
	{main.vhd}
	{Настройка главного модуля}

\section*{Задание 7}

В программе Xilinx PACE создать файл ограничений *.ucf или добавьте в проект имеющийся main\_xc3s200.ucf.
Исходный код представлен в листинге \ref{lst:main_xc3s200.ucf}

\includelisting
	{main_xc3s200.ucf}
	{Настройка портов}

\section*{Задание 8-9}
	В САПР ISE выполнить автоматический синтез технологической схемы, размещение и трассировку полученного устройства на кристалле Spartan XC3S 500E, генерировать файл конфигурации ПЛИС (*.bin).
	Выполнить программирование макетной ПЛИС Spartan3 отладочного набора Nexys2.
Провести тестирование разработанного устройства.
Тестирование было проведено успешно, устройство инкрементировало значение счетчика выводимого на дисплей по нажатию на кнопку, при этом отфильтровывая дребезг.

\section*{Вывод}
Были закреплены на практике знания полученных при изучении методики проектирования цифровых устройств на основе программируемых логических интегральных схем (ПЛИС). Были получены знания и навыки разработки устройства для подавления дребезга и работы с 7-сегментным индикатором (с применением динамической индикации).

\end{document}
\begin{comment}
На~рисунке~\ref{img:tux} символ семейства Unix-подобных операционных систем Linux.
Он отличается от~<<обычных>> пингвинов желтым цветом клюва и~лап.

\includeimage
    {tux} % Имя файла без расширения (файл должен быть расположен в директории inc/img/)
    {f} % Обтекание (без обтекания)
    {h} % Положение рисунка (см. figure из пакета float)
    {0.25\textwidth} % Ширина рисунка
    {Символ Linux (Tux)} % Подпись рисунка

На~листингах представлен исходный код программы Hello World на~языке программирования C в~двух вариантах оформления.

\includelisting
    {main.c} % Имя файла с расширением (файл должен быть расположен в директории inc/lst/)
    {Исходный код программы Hello World} % Подпись листинга

\includelistingpretty
    {main.c} % Имя файла с расширением (файл должен быть расположен в директории inc/lst/)
    {c} % Язык программирования (необязательный аргумент)
    {Исходный код программы Hello World} % Подпись листинга


        %\implicant{0}{2}
        %\implicant{5}{15}
        %\implicantedge{1}{3}{9}{11}
        %\implicantcorner
        %\implicantedge{4}{12}{6}{14}


\begin{center}
\begin{table}[ht]
	\caption{Карты карно для SN}
	\begin{tabular}{|c|c|c|c|c|}
		\hline
		\multicolumn{5}{| c |}{\textbf{SN(1)}} \\
		\hline
		\backslashbox[140px]{\scriptsize{S(1) S(0)}}{\scriptsize{COUNT DLY\_EN}} & 00 & 01 & 11 & 10\\
		\hline
		00 & 0 & 0 & 0 & 0\\
		\hline
		01 & 1 & 1 & 0 & 0\\
		\hline
		11 & 1 & 1 & 0 & 0\\
		\hline
		10 & 1 & 1 & 1 & 1\\
		\hline
	\end{tabular}
	\quad
	\begin{tabular}{|c|c|c|c|c|}
		\hline
		\multicolumn{5}{| c |}{\textbf{SN(0)}} \\
		\hline
		\backslashbox[140px]{\scriptsize{S(1) S(0)}}{\scriptsize{COUNT DLY\_EN}} & 00 & 01 & 11 & 10\\
		\hline
		00 & 0 & 1 & 1 & 0\\
		\hline
		01 & 0 & 0 & 1 & 1\\
		\hline
		11 & 1 & 1 & 0 & 0\\
		\hline
		10 & 0 & 1 & 1 & 0\\
		\hline
	\end{tabular}
\end{table}
\end{center}


\end{comment}
